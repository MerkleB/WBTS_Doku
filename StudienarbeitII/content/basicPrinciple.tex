% Da zu erwarten ist, dass nicht das komplette Konzept in dieser Studienarbeit
% realisiert werden kann, erfolgt die Entwicklung und Publikation mit einer
% OpenSource-Lizenz. Um die Weiterentwicklung für eine Community attraktiv zu
% machen, wurde mit Ruby on Rails ein Framework verwendet, welches sich im
% OpenSource-Bereich einer großen Beliebtheit erfreut.

% Meister, also Mitglieder mit bester didaktischer
% Qualifikation, werden Autoren von WBTs. Dadurch soll Qualität 

\chapter{Das zugrundeliegende Prinzip}
Ziel des Systems ist die Vermittlung von Lerninhalten in einer sich gegenseitig
Unterstützenden Gemeinschaft. Zu diesem Zweck folgt das Konzept einer Art
Mischung aus Lern- und Datingplattform -- es werden Lerninhalte bereitgestellt,
zu denen Tutoren vermittelt werden.

\section{Namensgebung}\label{ref:naming}
Der Name "`Masterly Mate"' entstand aus der Bezeichnung des höchsten Rangs im
Dreyfus-Modell (siehe Abschnitt \ref{ref:dreyfus}) und dem Namen eines
beliebten Getränks im Informatikerkreis, beziehungsweise dem englischen Begriff
für Kumpel oder Kamerad.

So lässt sich der Name frei als meisterlicher Kamerad übersetzen, was die
erwünschte offene und freundliche Kommunikation auf der Plattform ausdrücken
soll.

\section{Freie Software}\label{ref:freeLicensesConcept}
Um Offenheit gleich bei der Entwicklung zu berücksichtigen, wird Masterly Mate
unter einer freien Lizenz, wie sie in Abschnitt \ref{ref:freeLicenses}
beschrieben ist, veröffentlicht werden. Damit kann jeder interessierte den
Quelltext einsehen und bei Bedarf selbst Hand anlegen, um Funktionalitäten zu
verbessern oder neue hinzuzufügen. So gibt es auch keine Falltüren im Sinne von
ungewünscht übermittelten und verwendeten Informationen und damit fehlender
Transparenz, wie beispielsweise bei Facebook oder Google.

Auch das vorliegende Dokument wird unter einer freien Lizenz, der \ac{GFDL} zur
Verfügung gestellt. So befindet sich nach der Eigenständigkeitserklärung ein
Lizenzhinweis. Nach den Vorgaben wird im Anhang ab Seite \pageref{ref:gfdl} die
komplette Lizenz aufgeführt. Daraus folgt, dass jeder Interessierte das Projekt
und dessen Ursprünge verfolgen kann. Auch bietet das Dokument Einblicke in Ideen
für weitere Versionen in Abschnitt \ref{ref:weitereIdeen} und anschließende
Vorhaben in Abschnitt \ref{ref:anschlVorh}.


\begin{k}
Hier noch viel mehr beschreiben!

%\section{Bearbeiten von WBTs}
Ein Anwender, der eine fachliche Herausforderung sucht oder sich in einem Fach
weiterbilden möchte, wird sich dem Bearbeiten von WBTs widmen.


, indem er WBTs durcharbeitet und mit Bestehen
der darin enthaltenen Quizes Punkte für seinen fachlichen Rang sammelt. Unter
Umständen nimmer er Hilfe von einem Tutor in Anspruch

%\section{Tutorensuche}
Er erhält gegebenenfalls eine gute Bewertung und
bessert damit seinen didaktischen Rang auf.
\end{k}