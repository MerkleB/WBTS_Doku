% Da zu erwarten ist, dass nicht das komplette Konzept in dieser Studienarbeit
% realisiert werden kann, erfolgt die Entwicklung und Publikation mit einer
% OpenSource-Lizenz. Um die Weiterentwicklung für eine Community attraktiv zu
% machen, wurde mit Ruby on Rails ein Framework verwendet, welches sich im
% OpenSource-Bereich einer großen Beliebtheit erfreut.

% Meister, also Mitglieder mit bester didaktischer
% Qualifikation, werden Autoren von WBTs. Dadurch soll Qualität 

\chapter{Das zugrundeliegende Prinzip}
Ziel des Systems ist die Vermittlung von Lerninhalten in einer sich gegenseitig
Unterstützenden Gemeinschaft. Zu diesem Zweck folgt das Konzept einer Art
Mischung aus Lern- und Datingplattform -- es werden Lerninhalte bereitgestellt,
zu denen Tutoren vermittelt werden.

\section{Namensgebung}\label{ref:naming}
Der Name "`Masterly Mate"' entstand aus dem höchsten Rang im Dreyfus-Modell
(siehe Abschnitt \ref{ref:dreyfus}) und einem beliebten Getränk im
Informatikerkreis, beziehungsweise dem englischen Begriff für Kumpel/Kamerad.

So lässt sich der Name frei als meisterlicher Kamerad übersetzen, was die
erwünschte offene und freundliche Kommunikation auf der Plattform ausdrücken
soll.

\begin{k}
Hier noch viel mehr beschreiben!

Masterly Mate ist Freie Software\label{ref:freeLicensesConcept}
\ref{ref:freeLicenses}

noch Überraschungen mit einbringen? \cite{korte:2009}

%\section{Bearbeiten von WBTs}
Ein Anwender, der eine fachliche Herausforderung sucht oder sich in einem Fach
weiterbilden möchte, wird sich dem Bearbeiten von WBTs widmen.


, indem er WBTs durcharbeitet und mit Bestehen
der darin enthaltenen Quizes Punkte für seinen fachlichen Rang sammelt. Unter
Umständen nimmer er Hilfe von einem Tutor in Anspruch

%\section{Tutorensuche}
Er erhält gegebenenfalls eine gute Bewertung und
bessert damit seinen didaktischen Rang auf.
\end{k}