\chapter{Umsetzung}\label{ref:chaptImplementation}

\section{Nutzerverwaltung}
\begin{k}
Uebernimmt Julian?
\end{k}

\section{Lokationen}
\begin{k}
Uebernimmt Julian?
\end{k}

\section{Suche}
\begin{k}
Uebernimmt Julian/Benni?
\end{k}

\section{SCORM}\label{ref:implSCORM}
Die SCORM-Funktionalität wird in dem Objekt "`wbt"' realisiert, welches in
Abschnitt \ref{ref:objectWBT} beschrieben wurde. 

Die darin enthaltene Upload-Methode, die beim Einfügen und Ändern eines WBTs
aufgerufen wird, speichert das WBT auf dem lokalen Speicher des Servers. Dabei
wird das \ac{PIF} mithilfe der Funktionen aus einem
scorm-gem\footnote{Ressource: \url{https://rubygems.org/gems/scorm}} direkt
entpackt. Dabei ist mit der Berücksichtigung der Validierung stets ein Fehler
aufgetreten. Für die erste Version wird daher keine Validierung unterstützt, das
PIF wird ohne Prüfung entpackt (siehe Abschnitt \ref{ref:problems}). Die
Fehlerursache liegt unter Umständen an dem Alter des gems. Eine genauere Betrachtung ist für
spätere Versionen von Masterly Mate angedacht. 

Mit dem Hochladen und Entpacken werden die nötigen Attibute für den Start des
WBT gepflegt. Dies ist zum einen der Paketname selbst und zum anderen der Pfad
zur Start-Datei des Root SCO. In einer start-Methode werden diese Attribute
ausgelesen und das WBT wird in einem neuen Fenster geöffnet. So kann das WBT den
Raum einnehmen, den es braucht. Masterly Mate bleibt damit unabhängig vom Stil
des Autorenwerkzeugs. Für die erste produktive Version fehlt es noch an einem
geeigneten RTE, da dessen Implementierung den Rahmen der vorliegenden Arbeit
sprengen würde. Daher erfolgt die Registrierung des Ergebnisses zunächst durch
eine manuelle Eingabe des Nutzers.

\section{Lizensierung}
Nach den in Abschnitt \ref{ref:freeLicenses} beschriebenen freien Lizenzen und
dem Einbringen der Idee in das Konzept (siehe Abschnitt
\ref{ref:freeLicensesConcept}) wurde für die Zwecke von Masterly Mate auf die
\ac{AGPL} zurückgegriffen. Da diese mit der \ac{GPL} kompatibel ist, wird
eine eventuelle Verbreitung der Software im Sinne von OpenSource möglich.
Darüber hinaus kann das Projekt unter anderen kompatiblen Lizenten verbreitet
werden \cite{fsf:2007}.

Mit der Nutzung der AGPL entsteht die Pflicht, den Quelltext der Anwendung
direkt als Download anzubieten. Dazu wird im Interface Masterly Mate
ein Link auf die GIT-Ressource im Footer angeboten. Zusätzlich wurde, wie bei
allen Lizenzen nötig, in jeder Datei ein Lizenztext vorrangestellt.

\section{Dokumentation}
Wie in Abschnitt \ref{ref:archDoc} beschrieben, wurde mithilfe der Befehlszeile
\textit{rake doc:app} eine Dokumentation der API erstellt und im Footer
mit einem relativen Link auf \textit{doc/app/index.html} referenziert. Die im
Projekt verwendete Version 10.0.2 von rake bedient sich RDoc in der Version
2.12.2 und dem Darkfish Rdoc Generator 3.

Zusammen mit den Anmerkungen im Quelltext entsteht so eine ausführliche
Dokumentation der Fähigkeiten und Schnittstellen. In einem Index über die
Klassen und Module kann gezielt nach bestimmten Funktionsweisen gesucht werden.
Die dazu angebrachten Kurzbeschreibungen und Verweise bieten einen einfachen
Überblick und verhelfen einem interessierten Anwender oder Programmierer zur
Transparenz über die Funktionsweise von Masterly Mate.

Die Dokumentation der API ist in der englischen Sprache gehalten, da diese für
gewöhnlich nicht multilingual geführt wird und Englisch als Sprache für
computerrelevante Themen anerkannt ist.

\section{Gestaltung}
Das Design von Masterly Mate ist in der ersten Version zunächst einmal von
Funktionalität geprägt. Jedoch finden schon einige der weiter oben aufgezählten
gestalterischen Grundkonzepte hier Eingang. So ist es der Erwartungskonformität
geschuldet, dass Navigation und Anzeige sich in optisch voneinander getrennten
Bereichen befinden. Ebenfalls von vielen anderen Seiten bekannt, ist dass
Konzept, die Änderung der Sprach oben rechts und damit abseits von allen anderen
Kontrollen zu platzieren. Die Navigation ist von der Anzeige durch das Gesetz
der Nähe und einen Sichtbaren Trennstrich getrennt. Um eine bessere
Übersichtlichkeit auf den WBTs zu gewährleisten, werden diese außerdem in einem
neuen Tab geöffnet. Daten wie Themen, Benutzer oder WBTs werden in Tabellenform
präsentiert um dies übersichtlich zu gestalten. Die verwendete Schrift in
Masterly Mate ist Serifenlos, da dies für technische Zwecke sinnvoller ist. Da
Masterly Mate als gamifizierte Anwendung nicht vollkommen auf ein Mindestmaß an
ansprechender Optik verzichten kann, ist es in einem warmen Orange gehalten.
Die wengien anderen Farben sind so gewählt, dass sie nicht zu sehr in Kontrast
zur Hauptfarbe stehen (\cite{johnson:2000};\cite{nielsen:1995}).

Für die Zukunft wäre es denkbar, dem Benutzer Individualisierungsmöglichkeiten
zuzugestehen. Er könnte z.B. die Hintergrundfarbe bzw. das Thema mit dem
Masterly Mate dargestellt werden soll Ändern. Außerdem soll es möglich sein,
dass der Benutzer einen Avatar erhält. Eventuell könnte der Benutzer mit diesem
Avatar interagieren und von seinen erspielten Punkten Items für diesen erwerben. 

\section{Authentifizierung}
\begin{k}
Uebernimmt Julian?
\end{k}

\section{Themen}
Der Model Topic, welches die Themen abbildet verfügt über die Attribute name und
parent\_name. Zweiteres dient zur Identifizierung des Überthemas. Die Wahl
dieses Form der Referenz zu verwenden ist der Datenbank geschuldet. Auf diese Weise
liegt der Fremdschlüssel jeweils auf der n-Seite der 1:n Beziehung.

Wie schon erwähnt werden bei der Löschung eine Topic-Objektes alle
darunterliedenden Themen und WBTs in das Eltern-Topic des gelöschten Themen
geschoben. Damit bei einer Löschung das parent\_name-Attribut eines jeden
enthaltenen Themen geändert werden. Dies bedeutet außerdem, dass kein Thema ein
leeres parent\_name-Attribut enthält. Im Standartfall verweißt das
Attribut parent\_name auf das Root-Thema. Diese Lösung ist für Themen vollends
genügend und bedarf nach jetzigem Stand keiner Änderung. Dies steht im Gegensatz
zu den WBTs, bei welchen in Zukunft noch zu klären wäre, was mit solche
geschehen soll die im Root-Thema liegen. 