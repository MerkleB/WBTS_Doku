\chapter{Umsetzung}\label{ref:chaptImplementation}

\section{Nutzerverwaltung}
\begin{k}
Uebernimmt Julian?
\end{k}

\section{Lokationen}
\begin{k}
Uebernimmt Julian?
\end{k}

\section{Suche}
\begin{k}
Uebernimmt Julian/Benni?
\end{k}

\section{SCORM}\label{ref:implSCORM}
Die SCORM-Funktionalität wird in dem Objekt "`wbt"' realisiert, welches in
Abschnitt \ref{ref:objectWBT} beschrieben wurde. 

Die darin enthaltene Upload-Methode, die beim Einfügen und Ändern eines WBTs
aufgerufen wird, speichert das WBT auf dem lokalen Speicher des Servers. Dabei
wird das \ac{PIF} mithilfe der Funktionen aus einem
scorm-gem\footnote{Ressource: \url{https://rubygems.org/gems/scorm}} direkt
entpackt. Dabei ist mit der Berücksichtigung der Validierung stets ein Fehler
aufgetreten. Für die erste Version wird daher keine Validierung unterstützt, das
PIF wird ohne Prüfung entpackt (siehe \ref{ref:problems}). Die Fehlerursache
liegt unter Umständen an dem Alter des gems. Eine genauere Betrachtung ist für
spätere Versionen von Masterly Mate angedacht. 

Mit dem Hochladen und Entpacken werden die nötigen Attibute für den Start des
WBT gepflegt. Dies ist zum einen der Paketname selbst und zum anderen der Pfad
zur Start-Datei des Root SCO. In einer start-Methode werden diese Attribute
ausgelesen und das WBT wird in einem neuen Fenster geöffnet. So kann das WBT den
Raum einnehmen, den es braucht. Masterly Mate bleibt damit unabhängig vom Stil
des Autorenwerkzeugs. Für die erste produktive Version fehlt es noch an einem
geeigneten RTE, da dessen Implementierung den Rahmen der vorliegenden Arbeit
sprengen würde. Daher erfolgt die Registrierung des Ergebnisses zunächst durch
eine manuelle Eingabe des Nutzers.

\section{Lizensierung}
Nach den in Abschnitt \ref{ref:freeLicenses} beschriebenen freien Lizenzen und
dem Einbringen der Idee in das Konzept (siehe Abschnitt
\ref{freeLicensesConcept}) wurde für die Zwecke von Masterly Mate auf die
\ac{AGPL} zurückgegriffen. Da diese mit der \ac{GPL} kompatibel ist, wird
eine eventuelle Verbreitung der Software im Sinne von OpenSource möglich.
Darüber hinaus kann das Projekt unter anderen kompatiblen Lizenten verbreitet
werden \cite{fsf:2007}.

Mit der Nutzung der AGPL entsteht die Pflicht, den Quelltext der Anwendung
direkt als Download anzubieten. Dazu wird im Interface Masterly Mate
ein Link auf die GIT-Ressource im Footer angeboten. Zusätzlich wurde, wie bei
allen Lizenzen nötig, in jeder Datei ein Lizenztext vorrangestellt.

\section{Dokumentation}
\begin{k}
mit rake doc:app wurde eine API-Dokumentation erstellt, die im Footer
referenziert wurde

Uebernimmt Julian?
\end{k}

\section{Gestaltung}
\begin{k}
Das Design von Masterly Mate ist in der ersten Version zun�chst einmal von
Funktionalit�t gepr�gt. Jedoch finden schon einige der weiter oben aufgez�hlten
gestalterischen Grundkonzepte hier Eingang. So ist es der Erwartungskonformit�t
geschuldet, dass Navigation und Anzeige sich in optisch voneinander getrennten
Bereichen befinden. Ebenfalls von vielen anderen Seiten bekannt, ist dass
Konzept, die �nderung der Sprach oben rechts und damit abseits von allen anderen
Kontrollen zu platzieren. 
Die Navigation ist von der Anzeige durch das Gesetz der N�he und einen
Sichtbaren Trennstrich getrennt. 
Um eine bessere �bersichtlichkeit auf den WBTs zu gew�hrleisten, werden diese
Au�erdem in einem neuen Tab ge�ffnet. Daten wie Themen, Benutzer oder WBTs
werden in Tabellenform pr�sentiert um dies �bersichtlich zu gestalten. 
Die verwendete Schrift in Masterly Mate ist Serifenlos, da dies f�r Technische
Zwecke sinnvoller ist. Da Masterly Mate als gamifizierte Anwendung nicht
vollkommen auf ein Mindestma� an ansprechender Optik verzichten kann, ist es in einem warmen Orange gehalten.
Die wengien anderen Farben sind so gew�hlt, dass sie nicht zu sehr in Kontrast
zur Hauptfarbe stehen. 
F�r die Zukunft w�re es denkbar, dem Benutzer Individualisierungsm�glichkeiten
zuzugestehen. Er k�nnte z.B. die Hintergrundfarbe bzw. das Thema mit dem
Masterly Mate dargestellt werden soll �ndern. Au�erdem soll es m�glich sein,
dass der Benutzer einen Avatar erh�lt. Eventuell k�nnte der Benutzer mit diesem
Avatar interagieren und von seinen erspielten Punkten Items f�r diesen erwerben.
\end{k}

\section{Authentifizierung}
\begin{k}
Uebernimmt Julian?
\end{k}

\section{Themen}
\begin{k}
Der Model Topic, welches die Themen abbildet verf�gt �ber die Attribute name und
parent_name. Zweiteres dient zur Identifizierung des �berthemas. Die Wahl dieses
Form der Referenz zu verwenden ist der Datenbank geschuldet. Auf diese Weise
liegt der Fremdschl�ssel jeweils auf der n-Seite der 1:n Beziehung.
Wie schon erw�hnt werden bei der L�schung eine Topic-Objektes alle
darunterliedenden Themen und WBTs in das Eltern-Topic des gel�schten Themas
geschoben. Damit bei einer L�schung das parent_name-Attribut eines jeden
enthaltenen Themas ge�ndert werden. Dies bedeutet au�erdem, dass kein Thema ein
leeres parent_name-Attribut enth�lt. Im Standartfall verwei�t das
Attribut parent_name auf das Root-Thema. Diese L�sung ist f�r Themen vollends
gen�gens und bedarf nach jetzigem Stand keiner �nderung. Dies steht im Gegensatz
zu den WBTs, bei welchen in Zukunft noch zu kl�ren w�re, was mit solche
geschehen soll die im Root-Thema liegen.
\end{k}