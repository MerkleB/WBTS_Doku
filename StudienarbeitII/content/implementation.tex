\chapter{Umsetzung}\label{ref:chaptImplementation}

\section{Nutzerverwaltung}
\begin{k}
Uebernimmt Julian?
\end{k}

\section{Lokationen}
\begin{k}
Uebernimmt Julian?
\end{k}

\section{Suche}
\begin{k}
Uebernimmt Julian/Benni?
\end{k}

\section{SCORM}
Die SCORM-Funktionalität wird in dem Objekt "`wbt"' realisiert, welches in
Abschnitt \ref{ref:objectWBT} beschrieben wurde. 

Die darin enthaltene Upload-Methode, die beim Einfügen und Ändern eines WBTs
aufgerufen wird, speichert das WBT auf dem lokalen Speicher des Servers. Dabei
wird das \ac{PIF} mithilfe der Funktionen aus einem
scorm-gem\footnote{Ressource: \url{https://rubygems.org/gems/scorm}} direkt
entpackt. Dabei ist mit der Berücksichtigung der Validierung stets ein Fehler
aufgetreten. Für die erste Version wird daher keine Validierung unterstützt, das
PIF wird ohne Prüfung entpackt (siehe \ref{ref:problems}). Die Fehlerursache
liegt unter Umständen an dem Alter des gems. Eine genauere Betrachtung ist für
spätere Versionen von Masterly Mate angedacht. 

Mit dem Hochladen und Entpacken werden die nötigen Attibute für den Start des
WBT gepflegt. Dies ist zum einen der Paketname selbst und zum anderen der Pfad
zur Start-Datei des Root SCO. In einer start-Methode werden diese Attribute
ausgelesen und das WBT wird in einem neuen Fenster geöffnet. So kann das WBT den
Raum einnehmen, den es braucht. Masterly Mate bleibt damit unabhängig vom Stil
des Autorenwerkzeugs. Für die erste produktive Version fehlt es noch an einem
geeigneten RTE, da dessen Implementierung den Rahmen der vorliegenden Arbeit
sprengen würde. Daher erfolgt die Registrierung des Ergebnisses zunächst durch
eine manuelle Eingabe des Nutzers.

\section{Lizensierung}
Nach den in Abschnitt \ref{ref:freeLicenses} beschriebenen freien Lizenzen und
dem Einbringen der Idee in das Konzept (siehe Abschnitt
\ref{freeLicensesConcept}) wurde für die Zwecke von Masterly Mate auf die
\ac{AGPL} zurückgegriffen. Da diese mit der \ac{GPL} kompatibel ist, wird
eine eventuelle Verbreitung der Software im Sinne von OpenSource möglich.
Darüber hinaus kann das Projekt unter anderen kompatiblen Lizenten verbreitet
werden \cite{fsf:2007}.

Mit der Nutzung der AGPL entsteht die Pflicht, den Quelltext der Anwendung
direkt als Download anzubieten. Dazu wird im Interface Masterly Mate
ein Link auf die GIT-Ressource im Footer angeboten. Zusätzlich wurde, wie bei
allen Lizenzen nötig, in jeder Datei ein Lizenztext vorrangestellt.

\section{Dokumentation}
\begin{k}
mit rake doc:app wurde eine API-Dokumentation erstellt, die im Footer
referenziert wurde

Uebernimmt Julian?
\end{k}

\section{Gestaltung}
\begin{k}
Uebernimmt Benni?
\end{k}

\section{Authentifizierung}
\begin{k}
Uebernimmt Julian?
\end{k}

\section{Themen}
\begin{k}
Uebernimmt Benni
\end{k}