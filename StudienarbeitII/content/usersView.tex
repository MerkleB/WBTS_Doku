\chapter{Anwenderseite}
\section{Gamification}\label{ref:gamificationConcept}
In Abschnitt \ref{ref:gamification} wurden die wesentlichen Aspekte von
Gamification angesprochen. Hier erfolgt die Konkretisierung derartiger
Eigenschaften in die Konzeption von Masterly Mate.

\subsection{Belohnungskriterien}
Um die Motivation stets auf einem hohen Niveau zu halten, bedient sich Masterly
Mate einer Auswahl an Belohnungskriterien.

Die \textbf{erreichte Punktzahl} im Quiz eines WBTs wird für das Erreichen des
nächst höheren Levels hinzugenommen. Analog dazu erhalten Tutoren
\textbf{Sterne} für qualitativ hohe Unterweisungen. Zuzüglich zu den regulären
Punkten für die fachlichen und den tutoriellen Rang können \textbf{Wertmarken}
gesammelt werden mit denen einem Avatar oder dem \ac{GUI} mehr
Gestaltungsmöglichkeiten verliehen werden.

\subsection{Etappenweise Herausforderungen}
Ein weiteres Mittel, um die Motivation zu fördern sind Spielmechaniken, die in
die Anwendung eingebracht werden. Für diese ist es erforderlich, dass für jede
Art von Nutzer Anreize bestehen. 

Die drei Ränge Neuling, regulärer Nutzer und Enthusiast sind nicht zu
verwechseln mit den Rängen des Dreyfus-Modells aus Abschnitt
\ref{ref:dreyfusConcept}. Die hier betrachteten Ränge in Bezug auf Gamification
berücksichtigen die Handhabe der Software als solche. Ein Neuling verwendet die
Anwendung zum ersten mal oder bisher nur wenige male. Der reguläre Nutzer ist
ein Stammgast der Plattform, während der Enthusiast schier nicht schlafen kann,
ohne die Anwendung täglich verwendet zu haben.

Allen Nutzern ist gemeinsam, dass sie sich selbst gut evaluieren können. Sie
erhalten die Möglichkeit Statistiken einzusehen. Für die erste Version von
Masterly Mate ist eine Art Ladebalken vorgesehen, welcher je nach Füllstand
zeigt, wie groß die Erfüllung der erforderlichen Gesamtpunktzahl für einen Rang
ist. Über seine Wertung kann sich ein Nutzer stets im Feld aller Nutzer
einordnen. Dabei kann jeder für sich individuell entscheiden, ob es für ihn
wichtig ist, in die Top-Ten zu gelangen \cite{grubenMerkeBabics:2012}.

\subsubsection{Neuling}
Als erstmaliger Nutzer einer Plattform oder Anwendung benötigen Neulinge einen
vereinfachten Einstieg und eine leicht verständliche Anleitung. Auch muss die
GUI übersichtlich gestaltet sein, sodass nicht bereits nach wenigen Klicks
Frustration entsteht.

In Masterly Mate wird daher von Beginn an eine Möglichkeit zur
Internationalisierung (siehe Abschnitt \ref{ref:internationalisierung})
umgesetzt. So sieht der Anwender das Interface stets in seiner Sprache der Wahl
und stößt somit nicht auf Verständnisprobleme.

Weiterhin erhält ein neuer Nutzer analog seines fachlichen Rangs Zugang zu
einfachen WBTs, die leicht verständlich sind. Stößt er hier bereits auf
Probleme, so kann er sich einen Tutor zu Rate ziehen
\cite{grubenMerkeBabics:2012}.

Eine Anleitung im Sinne eines Handbuches ist für die erste produktive Version
nicht angedacht.
  
\subsubsection{Regulärer Nutzer}
Nutzer, die die Anwendung in moderater Weise nutzen, sind grundsätzlich
zufrieden mit dem, was ihnen geboten wird. Sie sind aber noch
begeisterungsfähig.

Ein solcher Stammgast kann in Masterly Mate auf noch höhere Ränge aufsteigen,
denn ihm fällt es leichter den in Abschnitt \ref{ref:rankTopic} angesprochenen
jährlichen Test zu bestehen. Als Tutor strebt er eventuell danach ein Meister zu
werden.

Um ein Erlebnis analog des in Abschnitt \ref{ref:basFlow} beschriebenen Flows zu
unterstützen, steigt auch das Niveau der zur Verfügung stehenden WBTs. So kann
der Stammnutzer bei Bedarf stets neue Herausforderungen suchen
\cite{grubenMerkeBabics:2012}.

\subsubsection{Enthusiast}
Ein Enthusiast wird alles daran setzen seinen Experten oder meisterlichen Rang
zu behalten. Er wird also weiterhin Unterweisungen geben und genießt sein
Privileg, WBTs editieren zu können. Je nach Interesse kann er seinen Meister
Rang dazu nutzen in andere Fachgebiete einzusteigen und dort für sämtliche
Mitglieder unterhalb seines zugehörigen fachlichen Rangs mit Rat zur Seite zu
stehen \cite{grubenMerkeBabics:2012}.

\section{Generieren von Motivation}
Zusammen mit der im vorangegangenen Abschnitt beschribenen Konzeption von
Gamification in Masterly Mate wird an dieser Stelle die Integration der in
Abschnitt \ref{ref:basMotivation} beschriebenen Motivation erläutert.

In diesem Abschnitt wird ein Blick auf die motivierenden Merkmale aus
Anwendersicht beschrieben, die im Wesentlichen bereits im vorangegangenen
Abschnitt für die Umsetzung von Gamification besprochen wurden. Dieser wird
separiert in intrisische und extrinsische Motivation dargestellt.

\subsection{Intrinsisch motivierende Aspekte}
Ein Nutzer von Masterly Mate strebt von sich aus nach Weiterbildung und damit
nach einem höheren fachlichen oder didaktischen Rang. Beinahe nebenbei steigt er
dafür immer tiefer in fachliche Themen ein. Für seine Leistungen möchte er
Feedback erhalten und mit gleichgesinnten reden oder helfen. 

Tutoren haben zusätzlich die Möglichkeit ihr Langzeitgedächtnis mithilfe von
immer wiederkehrenden Unterweisungen zu schulen. Damit zusammenhängend
trainieren sie sich in ihren didaktischen Fähigkeiten.
  
\subsection{Weitere extrinsisch motivierende Merkmale}
Zu den intrinsischen Motivatoren kommen weitere extrinsische Motivatoren. Dazu
gehören der Erhalt von Belohnungen, mehr Möglichkeiten in der Gestaltung des
Interfaces oder Avatars oder eine Bestenliste zum Leistungsvergleich mit anderen
Nutzern.

Macht man sich in der Plattform einen Namen als Tutor oder beteiligt sich an
regen Diskussionen im Forum, so erfährt man Lob und Kritik von gleichgesinnten.
Insgesamt lässt Masterly Mate eine persönliche Analyse in Form von Statistiken
zu.