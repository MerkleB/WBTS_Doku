\chapter{Ausblick}\label{ref:chaptSummary}
Dieses letzte Kapitel rundet die vorliegende Studienarbeit mit einem Blick in
die Zukunft ab. Es werden Ideen für weitere Versionen präsentiert interessante
Rückschlüsse für einen umfänglichen Umgang mit Masterly Mate vorgestellt.

\section{Ideen für weitere Versionen}\label{ref:weitereIdeen}
Es hat sich gezeigt, dass das in der Vorlesung Gamification entstandene Konzept
bei weitem nicht umgesetzt werden konnte. Allein die Realisierung der
grundlegenden Gedanken war in der gegebenen Zeit möglich. Daher folgt an dieser
Stelle eine Auflistung von Ideen für weitere Versionen von Masterly Mate.

\subsection{Für Anwender}
\begin{description}
\item[Liste absolvierter WBTs] Bisher gibt es für Lernende keine
direkte Möglichkeit auf absolvierte Tests in WBTs zurückzusehen. Mit einer
automatisch generierten Liste, die über die Navigation erreichbar ist, wird dies
möglich.
\item[Profilbild] Gewöhlich identifizieren sich die Nutzer einer Plattform
anhand ihres Avatars oder Profilbildes. In Masterly Mate muss diese Funktion
noch im Model für Nutzer hinzugefügt werden. Unter Umständen kann auch eine
Abfrage auf die bekannte Avatar-Plattform gravatar.com inkludiert werden.
\item[Jährliche Tests] Die tief im Konzept verwobenen jährlichen Tests fehlen in
der ersten Version von Masterly Mate noch. Dies ist auch darauf zurückzuführen,
dass bisher nur der Rahmen geschaffen wurde. Der Inhalt, wie beispielsweise
diverse WBTs, Themen und Tests fehlen für eine umfangreiche Erfahrung.
\item[Rangverlust für Tutoren] So fehlt im Zusammenhang mit dem
vorangegangenen Aspekt auch der Rangverlust für Tutoren bei einem Quartal
ohne gegebener Unterweisung.
\item[Quickstart] Neulinge haben es derzeit noch schwer. Es fehlt an
einem Quick-Start Guide, einer (Video-)Instruktion oder einem (Video-)Tutorial.
So wird der Einstieg erheblich vereinfacht.
\item[Gesamtranking] Mit einem Gesamtranking können sich
interessierte Lernende gegenüber Anderen vergleichen. Diese Übersicht
soll keinen Konkurrenzkampf auslösen und wurde daher bisher nicht
berücksichtigt. Die Art der Umsetzung erfordert eine ausgeklügelte
Konzeption.
\item[Open-ID] Um einen Login zu vereinfachen kann ein ID-Pool, wie
beispielsweise Open-ID abgefragt werden. So muss sich der Nutzer auf
Masterly Mate kein extra Konto einrichten. Mit einer ID, die er auf
diversen Plattformen nutzt kann er so auch auf Masterly Mate einfach
sein Passwort eingeben und ist ohne explizite Registrierung
eingeloggt.
\item[Newsletter] Ein persönlicher Newsletter kann an
interessierte Nutzer versendet werden. Dieser enthällt
persönliche Statistiken oder Informationen über neu
eingegangene oder geänderte WBTs. Die genaue Zusammenstellung kann im Rahmen
dessen Konzeption beschlossen werden.
\item[Realitätsnahe Tests] Mit Tests, die denen in Schulen oder Hochschulen
ähneln können sich Lernende in Prüfungssituationen üben. Bei eventuell
auftretenden können sie auf die bereits vermittelbaren Tutoren zurückgreifen. 
\end{description}

\subsection{Anwendungsintern}
\begin{description}
\item[Forum] In einem Forum können allgemeine Themen diskutiert werden. Hier
können neue Ideen für die Strukturierung von Masterly Mate entstehen oder über
alltägliches philosophiert werden.
\item[Mailerfunktionalität] Der Mailer ist ein sehr akutes Thema. Nutzer
erhalten so eine bestätigungs E-Mail für die Registrierung und andere relevante
Informationen.
\item[Refactorings] Für die bessere Umsetzung von DRY und KISS sollten
gelegentlich Refactorings angebracht werden. So wird die Programmcodequalität
verbessert und die Quelltexte werden einfacher lesbar und somit für eine
Community an Programmierern besser zugänglich.
\item[Tests] RoR bietet ein Unit-Test-Framework, welches aus Zeitgründen bisher
keine Verwendung fand. Tests sollten jedoch angebracht werden, um die
Korrektheit zu wahren und das YAGNI-Prinzip einfacher umzusetzen.
\item[Implementierung einer SCORM RTE und eines SCORM Players] Die vollständige
Implementierung von SCORM ist ein weiteres akutes Thema, welches in Angriff
genommen werden sollte. Analog der Arbeit aus \cite{mitter:2005} sollte eine
SCORM-RTE und je nach Bedarf ein SCORM-Player implementiert werden, der die
Kommunikation zwischen Masterly Mate und den WBTs gewährleistet und so die
Vergabe der Wertung automatisiert.
\end{description}

\subsection{Überarbeitungen für das Konzept}
\begin{description}
\item[Investierte Zeit] In Masterly Mate wird mit dem aktuell
realisierten Konzept nur der Erfolg bei absolvierten Tests in WBTs
und gegebenen Unterweisungen honoriert. Mit Berücksichtigung der bisher nicht
beachteten investierten Zeit kommt zusätzlich eine extrinsisch motivierende
Komponente hinzu. Dabei sollte jedoch wie in \cite{korte:2009} aufgegriffen
darauf geachtet werden, dass keine Übermotivation resultiert.
\item[Wertung für WBTs] Am Ende jedes WBTs ist es sinnvoll eine Wertung abgeben
zu können. Es werden Aussagen zur Qualität getroffen, wobei meisterliche
Tutoren angehalten werden eventuell auftretende Mängel zu beheben. Darüber
hinaus kann entschieden werden, ob das WBT zu einfach oder zu schwer war.
Tendiert eine Wertung zu stark in eine Richtung, wird das WBT dem am nächsten passenden Rang zugeordnet.
\item[Adaptierung von Lerninhalten] Eine wesentlich aufwändigere Idee
berücksichtigt die Adaptierung von Lerninhalten. Dabei wird der
Schwierigkeitsgrad dynamisch den Leistungen des Lernenden angepasst. Aus
\cite{knall:2005} können dazu Erkenntnisse gewonnen werden. Für Masterly Mate
bedeutet dies, dass für dieses Zweck eine große Menge an WBTs erforderlich ist,
um möglichst feine Abstufungen treffen zu können.
\item[Überraschungen] In \cite{korte:2009} wird davon gesprochen, die
Motivation anhand von Überraschungen hoch zu halten. Ist ein Schema einmal
erkannt und ausgereizt, so entstehen zwei gegensätzliche Alternativen.
Lernende wenden sich entweder gelangweilt ab oder sie können sich für das
Konzept begeistern und engagieren sich sehr stark weiter. So könnten
beispielsweise große Ausrückstungsgegenstände für einen Avatar oder ein
Oberflächendesign freigeschalten werden. In Fachkreisen wird hier von
Achievements gesprochen. Desweiteren könnte die dynamische Schwierigkeit wie bei
der Adaptierung von Lerninhalten für Überraschungen sorgen. Es könnten
beispielsweise Bonuspunkte mit dem Lösen von nebensächlichen Rätseln
gesammelt werden.
\item[Weiterentwicklung des Konzeptes zur Trennung der Ränge] Bislang sind nur
wenige klare Grenzen zwischen Rängen offensichtlich. Eine Überarbeitung sollte
dahingehend geschehen, dass sich beispielsweise ein Professioneller klar von
einem Erfahrenen abgrenzt. Dazu könnten beispielsweise eine Art "`Rang
Abschlusstests"' konzipiert werden.
\end{description}

\section{Anschließende Rückschlüsse}\label{ref:anschlVorh}
An ein Release von Masterly Mate sind einige Erwartungen geknüpft. Auch ist
bisher nicht viel über den Verwendungszweck der Applikation abzusehen.

So bleibt abzuwarten, ob eine DE, wie sie in Abschnitt \ref{ref:blendedLearning}
beschrieben ist, tatsächlich nur in Ausnahmefällen angewandt wird oder ob die
Plattform hauptsächlich als ein Forum genutzt wird, in dem kurze Antworten auf
kurze Fragen ohne eine Berücksichtigung der Persönlichkeit des Lernenden
Anwendung findet. Darüber hinaus ist heute nicht abzusehen, ob das Konzept
genügent Potential inne hat, um populär für Lernende und Tutoren, aber auch
für andere Entwickler zu werden. Viel wird dabei von einem gewissen Marketing
abhängig sein. Ohne, dass sich etwas herumspricht wird dieses im weiten Internet
nur mühsam tatsächlich bekannt.

Offen ist auch, ob eventuell neue Verwendungszwecke des Konzeptes entstehen oder
ob das freie Masterly Mate kopiert und auf mehreren Plattformen individualisiert
installiert wird, so wie es die AGPL unter anderem vorsieht.

Schließlich öffnet sich die Frage, ob mithilfe von Masterly Mate ein idealer
Lehrer gefunden und definiert werden kann. Mit ausreichend großer Verbreitung
treffen schließlich Lernende auf Tutoren aus aller Welt.