\chapter{Fazit und Ausblick}\label{ref:chaptSummary}

\section{Ideen für weitere Versionen}\label{ref:weitereIdeen}
\begin{k}
\begin{itemize}
  \item Nutzer kann eine Liste aller WBTs sehen, die er absolviert hat
  \item Forum
  \item Durchführung von Refactorings für DRY KISS
  \item Profilbild
  \item SCORM RTE implementieren
  \item Mailerfunktionalität
  \item Investierte Zeit mit in das Konzept für Belohnungen einbringen
  \item Am Ende jedes WBT eine Wertung abgeben (zu schwer/zu leicht) -> tendiert
  eine Wertung zu stark in eine Richtung, wird das WBT dem am nächsten passenden
  Rang zugeordnet
  \item Adaptierung von Lerninhalten, wie in \cite{knall:2005}?
  \item SCORM-Player \cite{mitter:2005}
\end{itemize}

\subsection{Gamification für Newbies}
\begin{itemize}
    \item Quick-Start Guide (Video-)Instruction, (Video-)Tutorial
    \item Statistical evaluation (Ranking)
    \item Unlock big equipments for the selected design
    \item Possibility of using a Open-ID
    \item Newsletter-Feature
    \item Self-assessment regarding to school grades
  \end{itemize}
  
 \subsection{Gamification für Regulars}
 \begin{itemize}
    \item Collections of Achievements according to the current progression
  \end{itemize}
  
 \subsection{Gamification für Enthusiasten}
 \begin{itemize}
    \item Levels
    \item Dynamic difficulty / i.e. riddles
  \end{itemize}
  
 \subsection{Weitere Möglichkeiten für Gamification}
\begin{itemize}
\item Design Selection (Tamagochi, Avatar)
  \item Assistance possible
  \item Progress bar
  \item status message
  \item discussion room
  \item class room
  \end{itemize}
  
wie schaut das Gamification-Konzept von MM aus? Irgendwie ist Fortschrittsbalken
und Statistik und Möglichkeit als Tutor zu gering bis Meister. Gut ist der
Anreiz, ab Meister WBTs einbringen und editieren zu können. Wie war das mit dem
Avatar oder der UI, die immer besser gestaltet werden kann?

\end{k}

\section{Anschließende Vorhaben}
\begin{k}
sehen, was wird
\begin{itemize}
  \item wirklich nur in Ausnahmefällen eine DE? (siehe
  \ref{ref:blendedLearning})
  \item (un)populär
  \item neue Verwendungszwecke
  \item nur eine Installation oder mehrere
\end{itemize}

kann mithilfe von Masterly Mate ein idealer Lehrer gefunden werden?
\end{k}