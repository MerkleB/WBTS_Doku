\chapter{Entwurf}\label{ref:chaptScript}
Der in diesem Kapitel beschriebene Entwurf zeigt konkret, wie das Konzept von
Masterly Mate umgesetzt werden wird. Hier werden Schemata und Architekturen
entwickelt, die im Kapitel \ref{ref:chaptImplementation} in Programmcode
umgesetzt werden.

\section{Realisierungsmethodik}
Nachdem in den vorangegangenen Kapiteln das Konzept von Masterly Mate erläutert
wurde, stellt sich die Frage nach einer geeigneten Möglichkeit zur Umsetzung. Da
die Anwendung stets verfügbar, leicht erreichbar, modular und einfach zu
verwalten sein soll, ist \ac{RoR} das Mittel der Wahl. Dieses bietet viele
interessante Features in Form von sogenannten gems, die dank einer regen
Community stets aktualisiert und erweitert werden. Zudem unterstützt es moderne
Programmierparadigmen, wie \ac{DRY} und \ac{KISS}. Dadurch bleibt die Anwendung
aus Sicht der Programmierer übersichtlich und erscheint sehr strukturiert. Das
das Framework der \ac{MVC}-Architektur folgt, schafft einen weiteren Grundstein
zur Trennung von Zuständigkeiten\footnote{bekannter unter "`separation of
concerns"'} und sorgt auch damit für Übersichtlichkeit. 

Weiterhin ist dieses Framework für die Weiterentwicklung im
OpenSource-Bereich prädistiniert, da damit bisher populäre
Webanwendungen, wie Twitter, realisiert wurden.

Darüber hinaus wird darauf geachtet, die Komponenten nach und nach nur dann zu
entwickeln, wenn sie tatsächlich gebraucht werden. Diese Vorgehensweise nach dem
\ac{YAGNI}-Prinzip beugt ein überlaufenes, unübersichtliches und schwer
zu wartendes Produkt vor.

\section{Internationalisierung}\label{ref:internationalisierung}
\begin{k}
Keine Internationalisierung in WBTs! (WBTs brauchen Attribut: Sprache)
Was ist die Default-Language? -> Sprache des Nutzers (User auch Attribut:
Muttersprache oder Herkunftsland)

Uebernimmt Julian?
\end{k}

\section{Beschreibung des Entwurfsklassendiagramms und
Use-Case}\label{ref:classModel}
\begin{k}
Uebernimmt Benni

bitte irgendwo den Anker \verb+\label{ref:objectWBT}+\label{ref:objectWBT}
anbringen (dort, wo die Klasse WBT erläutert wird)
\end{k}

\section{Komponenten}
\begin{k}
Struktur hier unklar\ldots noch warten auf die anderen Inhalte von JeyB und
Benni.
\end{k}
Prinzipiell ist Masterly Mate aus zwei Komplexen aufgebaut. Zum einen kann sich
ein Nutzer fachlich weiterbilden. Zum Anderen bietet ein Nutzer als Tutor seine
Hilfe für ein bestimmtes Fachgebiet an.

\subsection{Durcharbeiten von WBTs}

\subsection{Tutorensuche}

\subsection{Navigation, Impressum, Kontakt}
\begin{k}
Uebernimmt Julian
\end{k}

\subsection{Themen}
\begin{k}
Uebernimmt Benni?
\end{k}