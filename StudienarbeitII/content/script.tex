\chapter{Entwurf}\label{ref:chaptScript}
\begin{k}
Anwenden von DRY, KISS, YAGNI
\end{k}
\section{Verwendetes Framework}
Als Framework verwenden wir Ruby on Rails. Dieses bietet viele interessante
Features, die dank einer regen Community stets aktualisiert und erweitert
werden. Weiterhin ist dieses Framework für die Weiterentwicklung im
OpenSource-Bereich prädistiniert, da damit bisher einige populäre Webanwendungen
realisiert wurden.

\section{Internationalisierung}\label{ref:internationalisierung}
\begin{k}
Keine Internationalisierung in WBTs! (WBTs brauchen Attribut: Sprache)
Was ist die Default-Language? -> Sprache des Nutzers (User auch Attribut:
Muttersprache oder Herkunftsland)

Uebernimmt Julian?
\end{k}

\section{Beschreibung des Entwurfsklassendiagramms und
Use-Case}\label{ref:classModel}
\begin{k}
Uebernimmt Benni
\end{k}

\section{Aufbau}
Prinzipiell ist Masterly Mate aus zwei Komplexen aufgebaut. Zum einen kann sich
ein Nutzer fachlich weiterbilden. Zum Anderen bietet ein Nutzer als Tutor seine
Hilfe für ein bestimmtes Fachgebiet an.

\subsection{Bearbeiten von WBTs}
Ein Anwender, der eine fachliche Herausforderung sucht oder sich in einem Fach
weiterbilden möchte, wird sich dem Bearbeiten von WBTs widmen.

\begin{k}
, indem er WBTs durcharbeitet und mit Bestehen
der darin enthaltenen Quizes Punkte für seinen fachlichen Rang sammelt. Unter
Umständen nimmer er Hilfe von einem Tutor in Anspruch
\end{k}
\subsection{Tutorensuche}
\begin{k}
Er erhält gegebenenfalls eine gute Bewertung und
bessert damit seinen didaktischen Rang auf.
\end{k}

\subsection{Navigation, Impressum, Kontakt}
\begin{k}
Uebernimmt Julian
\end{k}

\subsection{Themen}
\begin{k}
Uebernimmt Benni?
\end{k}