\chapter{Zusammenfassung der Ergebnisse}\label{ref:chaptConclusion}
Es hat sich gezeigt, dass \ac{RoR} sehr mächtig ist. Innerhalb kürzester Zeit
entstehen Webapplikationen mit umfangreichen Funktionen. Für Masterly Mate
reichten vorhandene gems aus und es musste kein proprietärer Ruby-Code
geschrieben werden.

\paragraph{Aufgetretene Probleme}\label{ref:problems}
Sehr problematisch war hingegen der Umgang mit SCORM. Diese, recht komplexe,
Spezifikation konnte längst nicht in vollem Umfang implementiert werden. Es
steht auch zur Diskussion, ob das gewählte scorm-gem die gewünschten Ziele in
Kommunikation mit einer SCORM-RTE fehlerfrei erfüllt. Derzeit entstehen, wie
bereits in Abschnitt \ref{ref:implSCORM} erwähnt, Fehler bei der Validierung von
PIFs.

Es war für die erste produktive Version angedacht, eine Kommunikation zwischen
\ac{LMS} und \ac{WBT} bereitzustellen. Nach Recherchen hat sich gezeigt, dass
dieses Vorhaben den Umfang eines eigenen Projektes beansprucht. Alternativ ist
die Verwendung des SCORM-Players, wie in \cite{mitter:2005} und
\cite{knall:2005} zu überdenken.

\begin{k}
ist das alles?
\end{k}