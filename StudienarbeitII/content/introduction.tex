\chapter{Einleitung}\label{ref:chaptIntroduction}
Basierend auf der Studienarbeit "`Analyse von Authorensystemen für ein WBT zu
Vorlesungszwecken von Michael Gruben \cite{gruben:2012} wird in dieser
Studienarbeit ein System aus \ac{WBT}s geschaffen. Das Konzept für das
Produkt des Projektes ist im Rahmen der Vorlesung "`Gamification"' entstanden.

Dabei handelt es sich grundsätzlich um eine Blended Learning Plattform, die
interessierten Lernenden eine zentrale Anlaufstelle bietet. Es werden also
eLearning und persönliches Lernen miteinander kombiniert. Umrahmt und
gamifiziert wird die Idee mithilfe des Dreyfus fünf Etappen Modells mentaler
Aktivitäten. Die in dieser Studienarbeit verwendeten Bezeichnungen unterliegen
gegebenenfalls weiteren Änderungen und sind für die deutschsprachige Version der
Plattform bestimmt.

Inhalte der vorligenden Studienarbeit sind Einblicke in die Entwicklung des
ersten Prototyps. Dazu zeigt Kapitel \ref{ref:chaptConcept} die Konzeption und
damit die grundlegende Idee der Architektur. Daran anschließend wird in Kapitel
\ref{ref:chaptScript} näher auf den tatsächlichen Entwurf eingegangen. Hier
wird konkret auf Klassen und Methoden eingegangen, welche die Realisierung
bestimmter Use-Cases zum Ziel haben. Kapitel \ref{ref:chaptImplementation}
zeigt, wie der Entwurf letztlich realisiert wird. Hier sind auch erste
Screenshots der Anwendung zu sehen. Um die vorrangegangenen Schritte
zusammenzufassen und kurz auszuwerten, gibt Kapitel \ref{ref:chaptConclusion}
einen Gesamtüberblick. Darauf aufbauend bietet Kapitel \ref{ref:chaptSummary}
eine Auswertung, die alle Aspekte des Projekts umfasst. Zusätzlich werden hier
Ausblicke auf die weitere Verwendung des Projektergebnisses gegeben.

Am Ende des Projekts steht ein funktionierender Prototyp, der die
wesentlichen Funktionen beherrscht. Weiterhin wird ein Konzept entwickelt worden
sein, welches das Projekt an zentralen Stellen bekannt macht und so für eine
rege Beteiligung sorgen soll. Mit der Namensgebung "`Masterly Mate"' wurde
bereits vor dem eigentlichen Projektstart ein wesentlicher Schritt zur
Bekanntmachung getan.