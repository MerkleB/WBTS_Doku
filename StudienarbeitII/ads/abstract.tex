\pagestyle{empty}

\renewcommand{\abstractname}{Zusammenfassung}

\begin{abstract}
Im Verlauf der Bearbeitung der Studienarbeit "`Analyse und Vergleich von
Autorensystemen für ein WBT zu Vorlesungsinhalten"' ist in der Vorlesung
"`Gamification"' ein Konzept für ein WBT\footnote{Web Based Training}-System
entstanden. Aus dieser Vorstellung ist die Idee, und damit die Motivation der
Studienarbeit, entstanden es in die Realität umzusetzen.

Es handelt sich um eine Webapplikation, die diverse WBTs in entsprechenden
Kategorien zum Bearbeiten anbietet. Das Lernmodell der Gebrüder Dreyfuß
wird in diese verwoben. In dem Modell wird die Kompetenz in einem Fachgebiet auf
zwei unterschiedlichen Ebenen betrachtet, die fachliche Kompetenz und die
Fähigkeit erklären zu können.

Zunächst wird die fachliche Kompetenz betrachtet. Demnach bearbeitet ein Neuling
auf dem ersten Kompetenzlevel eines bestimmten Fachbereiches ein grundlegendes
WBT, dessen abschließende Fragen nach vorgegebenen Schemata und grundlegender
Eigenschaften beantwortet werden. Ein Experte auf dem vierten Kompetenzlevel
muss hingegen Antworten auf Fragen wissen, die ein wesentlich komplexeres
Verständnis eines Sachverhaltes verlangen.

Um seine Fähigkeit erklären zu können unter Beweis zu stellen, engagiert man
sich mit Hilfestellungen für niedere fachliche Level. Beurteilen diese die
Hilfestellung als gut, kann der Mastery Rang erreicht werden, der sich noch über
dem Experten befindet. Nach dem Dreyfuß-Modell dürfen sich Lernender und
Lehrender durch maximal zwei Level unterscheiden. Der Mastery-Level ist hingegen
ein "`erklärender Experte"', der nicht nur fachlich höchst Kompetent ist,
sondern auch sehr gut auch für einen Anfänger erklären kann, ohne in fachliche
Details abzuschweifen.

Das WBT-System, welches beide beschriebenen Ebenen der Kompetenz organisiert,
wird unter einer freien Lizenz veröffentlicht werden. So kann das als noch sehr
simpel und eingeschränkt erwartete Ergebnis der Studienarbeit als Community
Projekt weiterleben und weiterentwickelt werden. Bereits vor Bearbeiten der
Studienarbeit wird damit gerechnet, das nur ein kleiner und spezieller aber
funktionaler Teil des Konzeptes umgesetzt werden wird. Der Fokus liegt
dabei grundsätzlich mehr auf Funktionalität, einer leicht zu erweiternden
Architektur der Software und einem benutzerfreundlichem Interface, als auf
einem gut aussehendem Design.
\end{abstract}
% \begin{abstract}
% Ein Abstract ist eine prägnante Inhaltsangabe, ein Abriss ohne
% Interpretation und Wertung einer wissenschaftlichen Arbeit. In DIN
% 1426 wird das (oder auch der) Abstract als Kurzreferat zur
% Inhaltsangabe beschrieben.
% 
% \begin{description}
% \item[Objektivität] soll sich jeder persönlichen Wertung enthalten
% \item[Kürze] soll so kurz wie möglich sein
% \item[Genauigkeit] soll genau die Inhalte und die Meinung der Originalarbeit wiedergeben
% \end{description}
% 
% Üblicherweise müssen wissenschaftliche Artikel einen Abstract
% enthalten, typischerweise von 100-150 Wörtern, ohne Bilder und
% Literaturzitate und in einem Absatz.
% 
% Quelle \url{http://de.wikipedia.org/wiki/Abstract} Abgerufen 07.07.2011
% \end{abstract}
% 
% 
% \renewcommand{\abstractname}{Summary}
% \begin{abstract}
% An abstract is a brief summary of a research article, thesis, review,
% conference proceeding or any in-depth analysis of a particular subject
% or discipline, and is often used to help the reader quickly ascertain
% the paper's purpose. When used, an abstract always appears at the
% beginning of a manuscript, acting as the point-of-entry for any given
% scientific paper or patent application. Abstracting and indexing
% services for various academic disciplines are aimed at compiling a
% body of literature for that particular subject.
% 
% The terms précis or synopsis are used in some publications to refer to
% the same thing that other publications might call an "abstract". In
% management reports, an executive summary usually contains more
% information (and often more sensitive information) than the abstract
% does.
% 
% Quelle: \url{http://en.wikipedia.org/wiki/Abstract_(summary)}

% \end{abstract}
